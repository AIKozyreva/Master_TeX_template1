%%% Литературный обзор %%%
\section{Литературный обзор}

\subsection{Подраздел 1}    
    
    Генетический материал холерного вибриона состоит из двух хромосом, может содержать плазмиды. Суммарный размер хромосом оценивается в 4 миллиона пар оснований. Обе хромосомы кодируют элементы, необходимые для жизнедеятельности клетки, например рРНК опероны, имеют постоянный размер и структуру, встречаются во всех представителях таксона.  
    
\subsection{Подраздел 2}

    Всемирная организация здравоохранения рекомендовала использовать антибиотические средства для лечения только тяжелых случаев холеры, и в современной стратегии борьбы с заболеванием массовое использование антибиотиков не рекомендуется \cite{book_sample}. Тем не менее, в ряде случаев антибиотики продемонстрировали свою эффективность, а потому стали применяться, наряду с регидратационными растворами, как основное средство в борьбе этим заболеванием \cite{repWHO2025}. Неизбежным итогом широкого применения антибиотиков против бактериальных инфекционных агентов становится эволюция и приспособление последних к применяемым против них препаратам \cite{CDC_AR_2019}. 

\subsection{Подраздел 3}
    
    Применение бактериофагов в медицине известно с начала XX века. С момента их открытия фаги успешно использовались для терапии инфекций, вызванных различными патогенными бактериями, включая \textit{Staphylococcus spp.}, \textit{Streptococcus spp.} и \textit{Pseudomonas aeruginosa}. Эффективность бактериофаги продемонстрировали против возбудителей дизентерии (\textit{Shigella spp.}), холеры (\textit{Vibrio cholerae}), брюшного тифа (\textit{Salmonella typhi}), а также для лечения раневых инфекций в военно-полевых условиях \cite{article_sample1}. 

    \subsubsection{Секция 1 внутри подраздела 3}
    
     Однако, широкое применение такого подхода ограничивается недостаточной изученностью молекулярных механизмов работы ряда антифаговых систем. 
    
\subsection{Подраздел 4}
    \subsubsection{Секция 1 внутри подраздела 4} 
    
    Описанный подход подразумевает цикл разнообразных экспериментальных и вычислительных работ. 

    \subsubsection{Секция 2 внутри подраздела 4}

    Вычисление CAI включает формирование эталонного набора генов. 

 %   Добавим несколько картинок (Рисунок \ref{fig:01-example-1}). 
 %   
 %   \addtwoimghere{01-example-1.png}{01-example-2.png}{0.49}{0.49}{В картинках также работают ссылки. Пусть \cite{newton2014newton}.}{fig:01-example-1}
 %   
 %   На приложение в тексте обязательно должна быть сделана ссылка ---  \hyperlink{app-a}{Приложение А}.
    
% Обязательно добавляем это в конце каждой секции, чтобы 
% обеспечить переход на новую страницу
\clearpage