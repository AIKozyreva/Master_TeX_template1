%%% Результаты %%%
% Preamble file

\section{Результаты}

\subsection{Сборка генома}
    В работу были переданы 5 штаммов, культивированные в течение 18 часов на немодифицированной среде LB при 37°C. В результате по данным секвенирования с использованием ONT-прочтений были получены сборки последовательностей геномов, что согласуется с данными литературы (Таблица \ref{table:01-genome_stat}). 
    
        \begin{table}[H]
        \caption{Результаты сборки геномов} \label{table:01-genome_stat}
        \begin{tabular}{|p{2.8cm}|p{2.8cm}|p{2.8cm}|p{2.8cm}|p{2.8cm}|}
        \hline Наименование & Суммарная длина, bp & Хромосома 1, bp & Хромосома 2, bp & Плазмида, bp \\
        \hline S1 & 4 109 167 & 2 980 553 & 1 129 214 & - \\
        \hline S2 & 4 938 774 & 2 912 449 & 1 191 583 & 834 742 \\
        \hline S3 & 4 064 585 & 2 993 592 & 1 070 993 & - \\
        \hline S4 & 4 113 653 & 3 053 066 & 1 060 587 & - \\
        \hline S5 & 4 091 768 & 3 042 768 & 1 049 000 & - \\
        \hline
        \end{tabular}
        \end{table}
    
    Каждая сборка состоит из двух или трёх замкнутых кольцевых последовательностей (Рисунок \ref{fig:assemblies_fig}) и не содержит крупных недосеквенированных участков генома, что подтверждается данными проверки целостности сборки с использованием BUSCO - количество обнаруженных в полном объёме групп ортологов составляет от 99.5 \%. 

    \begin{figure}
        \centering
        \includegraphics[width=1\linewidth]{pattern//img/01-example-1.png}
        \caption{Состав собранных геномов}
        \label{fig:assemblies_fig}
    \end{figure}

    \subsubsection{Секция 1 Подраздела Сборка генома}
        Для более полной картины сравнения следует отметить, ген \textit{flgN}, в референсе не представленный. 

%    Добавим несколько картинок (Рисунок \ref{fig:01-example-1}). 
%    \addtwoimghere{01-example-1.png}{01-example-2.png}{0.49}{0.49}{В картинках также работают ссылки. Пусть \cite{newton2014newton}.}{fig:01-example-1} 
%    На приложение в тексте обязательно должна быть сделана ссылка ---  \hyperlink{app-a}{Приложение А}.

% Обязательно добавляем это в конце каждой секции, чтобы 
% обеспечить переход на новую страницу
\clearpage
