%%% Выводы %%%
\section{Выводы}
% Сторонние функции для экспрессирующихся афз по данным литературы
\subsection{Геномная организация}
    
    Геномы большинства исследованных штаммов имеют характерную для вида организацию (2 кольцевые хромосомы) и суммарную длину около 4 млн пар оснований, что полностью соответствует современным представлениям о структуре генома этой бактерии. 
    
\subsection{Генетическое типирование}

    Типирование по ключевым генам полностью соответствует данным литературы и указывает на присутствие в выборке разнообразных по геномному составу вариантов патогена, характерных для разных волн пандемии. 

\subsection{Общие выводы и перспективы}

    Полученные результаты демонстрируют высокую степень геномного и функционального разнообразия штаммов как из клинических, так и из природных источников. Это подтверждает важность комплексного подхода к изучению эпидемиологии, эволюции и механизмов защиты возбудителя. 
    Полученные данные могут быть использованы для дальнейших исследований по микробиологии.
    
% Обязательно добавляем это в конце каждой секции, чтобы 
% обеспечить переход на новую страницу
\clearpage