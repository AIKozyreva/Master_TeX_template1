%%% Материалы и методы %%%
\section{Материалы и методы}
    В рамках экспериментальной части работы было проведено выращивание чистой культуры, тотальное выделение ДНК.
    
\subsection{Микробиология}
     Для культивирования и проведения дальнейшей работы были выбраны 5 штаммов. 
    
     \begin{table}[H]
           \caption{Штаммы} \label{table:01-strains}
           \begin{tabular}{|p{2.8cm}|p{1.8cm}|p{3.2cm}|p{3.8cm}|p{3cm}|}
           \hline Наименование & Год & Место \\
           \hline АФФ & 2011 & Калмыкия \\
           \hline МАА & 2016 & Москва \\
           \hline ИМА & 1970 & Саратов  \\
           \hline
           \end{tabular}
    \end{table}
     
     Выращивание чистой культуры проводили с использованием твёрдой агаризованной не модифицированной среды LB при температуре 37°C, pH=7.4 в течение 18 часов.  

\subsection{Геномика}

    Выделение ДНК проводили стандартным методом \cite{my20093}. 
    
    Подготовка библиотек геномной ДНК бактерий для секвенирования осуществлялась по протоколу Rapid Barcoding Kit 96 V14 (SQK-RBK114.96) с использованием соответствующих реагентов, рекомендованных в протоколе. Секвенирование геномной ДНК проводилось с использованием прибора PromethION компании Oxford Nanopore Technologies. 

\subsection{Анализ данных}

    Оценка качества полученных в результате секвенирования прочтений проводилась с использованием fastQC (v0.12.1) \cite{fastQC_2010}, Nanoplot (v1.42.0) \cite{nanoplot2023}, pycoQC (v2.5.0.3) \cite{pycoQC2019}. 
    
% \subsection{Первый раздел}
%   Сюда добавим какую-нибудь таблицу (Таблица \ref{table:01-coeffs}). 
%   
%   \begin{table}[H]
%       \caption{Таблица коэффициентов} \label{table:01-coeffs}
%       \begin{tabular}{|p{0.6cm}|p{4.9cm}|p{4.5cm}|p{4cm}|}
%       \hline \# & Колонка 1 & Колонка 2 & Колонка 3 \\
%       \hline 1 & Один & $f(x) + c$ & $4.1 $ \\
%       \hline 2 & Два & $f(x) - a$ & $4.2 $ \\
%       \hline 3 & Три & $f(x) \ \sim \ b$ & $4.3$ \\
%       \hline
%       \end{tabular}
%   \end{table}
%   
%   \subsubsection{Первый подраздел}
%       Первый подраздел первый подраздел первый подраздел первый подраздел первый подраздел первый подраздел первый подраздел первый подраздел первый подраздел первый подраздел первый подраздел первый подраздел первый подраздел 
%

% Обязательно добавляем это в конце каждой секции, чтобы 
% обеспечить переход на новую страницу
\clearpage