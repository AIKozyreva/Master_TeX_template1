%%% Список используемых сокращений %%%
\anonsection{Перечень сокращений и условных обозначений}

\begin{enumerate}[label={}]
    \item \hypertarget{АФЗ}{АФЗ --- Antiphage defense systems, Антифаговые системы защиты}  
    \item \hypertarget{BREX}{BREX --- Bacteriophage Exclusion system, антифаговая система с эпигенетическим механизмом защиты}
    \item \hypertarget{CAI}{CAI --- Codon Adaptation Index, индекс адаптации кодонов}
    \item \hypertarget{CBASS}{CBASS --- Cyclic-oligonucleotide-based antiphage signaling system, антифаговая система с использованием циклических сигнальных молекул}  
    \item \hypertarget{CDS}{CDS --- Coding DNA Sequence, Кодирующая последовательность}
    \item \hypertarget{dGTP}{dGTP --- дезоксигуанозинтрифосфат,  нуклеозидтрифосфат и предшественник нуклеотидов, используемый в клетках для синтеза ДНК}
    \item \hypertarget{dGTPase}{dGTPase --- фермент, гидролизующий dGTP, регулирующий пул нуклеотидов}   
    \item \hypertarget{ICE}{ICE --- Integrative and Conjugative Element, интегрируемый конъюгативный элемент} 
    \item \hypertarget{IS}{IS --- Insertion Sequence, вставочный элемент}  
    \item \hypertarget{Menshen}{Menshen --- антифаговая система, связанная с регуляцией клеточного стресса}   
    \item \hypertarget{PLE}{PLE --- Phage-inducible Chromosomal Island-like Elements, фагоиндуцируемые острова}  
    \item \hypertarget{PDC}{PDC-системы --- предсказанные антифаговые комплексы с неподтверждённой экспериментально функцией}  
    \item \hypertarget{RM}{RM-система --- Restriction-Modification system, система рестрикции-модификации}  
    \item \hypertarget{Retron}{Retron-I --- ретронная антифаговая система, синтезирующая msDNA}  
    \item \hypertarget{septu}{Septu --- антифаговая система, выявленная в нетоксигенных штаммах}  
    \item \hypertarget{tcpA}{\textit{tcpA} --- ген, кодирующий субъединицу токсин-корегулируемых пилей}  
    \item \hypertarget{vps}{\textit{vps} --- кластер генов, ответственных за биосинтез виброполисахарида}  
    \item \hypertarget{hapR}{\textit{hapR} --- регулятор quorum sensing, влияющий на синтез ЭПС и формирование ругозного фенотипа}
    \item \hypertarget{ICE_SXT}{SXT/R391 --- семейство интегрированных конъюгативных элементов, часто несущих гены резистентности}    
    \item \hypertarget{ctxB}{\textit{ctxB} --- ген субъединицы B холерного токсина}
    \item \hypertarget{Жгутиковые_кластеры}{Жгутиковые кластеры Flg и Fli --- генные кластеры, кодирующие белки жгутикового аппарата}
    \item \hypertarget{ЛПС}{ЛПС --- липополисахарид, компонент внешней мембраны грамотрицательных бактерий}
    \item \hypertarget{O-антиген}{O-антиген --- полисахаридный антиген, компонент липополисахарида, определяющий серогруппу}
    \item \hypertarget{Дефенсом}{Дефенсом --- Defensome, совокупность систем антифаговой защиты организма}
    \item \hypertarget{ДНК}{ДНК --- дезоксирибонуклеиновая кислота}
    \item \hypertarget{РНК}{РНК --- рибонуклеиновая кислота}
    \item \hypertarget{Транзиция}{Транзиция --- точечная мутация, обусловленная заменой одного пуринового основания на другое (аденина на гуанин или наоборот) или одного пиримидинового основания на другое (тимина на цитозин или наоборот)}
    \item \hypertarget{Трансверсия}{Трансверсия --- точечная мутация, обусловленная заменой пуринового основания (аденин, гуанин) на пиримидиновое (тимин, цитозин) и наоборот}
    \item \hypertarget{рРНК,тРНК}{рРНК, тРНК --- рибосомная РНК, транспортная РНК}
    \item \hypertarget{ЭПС}{ЭПС --- экзополисахарид, виброполисахарид}
    \item \hypertarget{R-форма}{R-форма --- форма липополисахарида с отсутствующим или сокращённым O-антигеном}
    \item \hypertarget{Эффлюксный_насос}{Эффлюксный насос --- белок, обеспечивающий активный вынос токсинов и антибиотиков из клетки}  
    \item \hypertarget{Фаг}{Фаг --- бактериофаг, вирус, инфицирующий бактерии}  
    \item \hypertarget{Протеомика}{Протеомика --- набор методов и технологий определения и анализа белков}  
    \item \hypertarget{Геномика}{Геномика --- совокупность методов исследования геномной структуры и функций организма}
    \item \hypertarget{Норма}{Норма, нормальные условия, стандартные условия --- в данной работе - условия культивирования бактерий: немодифицированная среда LB, 37°C, в течение 18 часов, в условиях отсутствия специфических стрессовых факторов, в частности контакта с бактериофагами}

\end{enumerate}

% Обязательно добавляем это в конце каждой секции, чтобы 
% обеспечить переход на новую страницу
\clearpage