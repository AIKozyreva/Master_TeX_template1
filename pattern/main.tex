%%% Если работа выполняется в Overleaf, 
%%% важно поменять тип компилятора
%%% с "pdfLatex" или "Latex" на "XeLaTeX". 
%%% Если этого не сделать, то и и компиляция

%%% Если диплом получается очень маленький, 
%%% смело увеличивай шрифт до 14го
\documentclass[xetex,a4paper,12pt]{article} % 12й шрифт
% Подключение преамбулы с необходимыми пакетами
\input{pattern/inc/preamble}

\begin{document}
    %%% Аннотацию добавляешь в папку inc. 
    %%% Когда будет готово, включишь эту строчку. 
    \includepdf[]{pattern/inc/Title_list_Masterthesis_Козырева_А.И.pdf} % Титульная страница
    \includepdf[pages=-]{pattern/inc/Zadanie_Kozyreva.pdf} % Титульная страница
    \includepdf[pages=-]{pattern/inc/Annotation_Masterthesis_Козырева_А.И.pdf} % Аннотация все страницы
    
    %%% В наш год положение о ВКР требовала именно такой 
    %%% последовательности секций работы.
    %%%% Задание ВКР %%%
\assignment{Задание ВКР}

Тема магистерской выпускной квалификационной работы --- <<XXXXXXXXXXXXXXXXXXXXXXXXXXXXXXXXXXX>>.

Задание ВКР: В выпускной квалификационной работе проводится анализ XXXXXXXXXXXXXXXXXXXXXXXXXXXXXXXXXXXXXXXXXXXXXXXXXXXXXX.  

В рамках работы были выполнены: XXXXX, XXXXXXXXXX, XXXXXXXXXXXXXXXXXXXXXXXXXXXXXXX.  

\clearpage % Задание ВКР
    %%% Аннотация %%%
\anonsection{О работе}

В работе представлен XXXXXXXXXXXXXXXXXXXXXXXX.  

Выпускная квалификационная работа магистра изложена на \pageref{LastPage} листах, включает X таблиц, X рисунков, X приложения, X литературных источников.

Ключевые слова: микробиология, медицинская микробиология, биобезопасность, секвенирование третьего поколения, ONT.

% Обязательно добавляем это в конце каждой секции, чтобы 
% обеспечить переход на новую страницу
\clearpage % Аннотация
    \tableofcontents % Содержание 
    \clearpage
    %%% Список используемых сокращений %%%
\anonsection{Перечень сокращений и условных обозначений}

\begin{enumerate}[label={}]
    \item \hypertarget{АФЗ}{АФЗ --- Antiphage defense systems, Антифаговые системы защиты}  
    \item \hypertarget{BREX}{BREX --- Bacteriophage Exclusion system, антифаговая система с эпигенетическим механизмом защиты}
    \item \hypertarget{CAI}{CAI --- Codon Adaptation Index, индекс адаптации кодонов}
    \item \hypertarget{CBASS}{CBASS --- Cyclic-oligonucleotide-based antiphage signaling system, антифаговая система с использованием циклических сигнальных молекул}  
    \item \hypertarget{CDS}{CDS --- Coding DNA Sequence, Кодирующая последовательность}
    \item \hypertarget{dGTP}{dGTP --- дезоксигуанозинтрифосфат,  нуклеозидтрифосфат и предшественник нуклеотидов, используемый в клетках для синтеза ДНК}
    \item \hypertarget{dGTPase}{dGTPase --- фермент, гидролизующий dGTP, регулирующий пул нуклеотидов}   
    \item \hypertarget{ICE}{ICE --- Integrative and Conjugative Element, интегрируемый конъюгативный элемент} 
    \item \hypertarget{IS}{IS --- Insertion Sequence, вставочный элемент}  
    \item \hypertarget{Menshen}{Menshen --- антифаговая система, связанная с регуляцией клеточного стресса}   
    \item \hypertarget{PLE}{PLE --- Phage-inducible Chromosomal Island-like Elements, фагоиндуцируемые острова}  
    \item \hypertarget{PDC}{PDC-системы --- предсказанные антифаговые комплексы с неподтверждённой экспериментально функцией}  
    \item \hypertarget{RM}{RM-система --- Restriction-Modification system, система рестрикции-модификации}  
    \item \hypertarget{Retron}{Retron-I --- ретронная антифаговая система, синтезирующая msDNA}  
    \item \hypertarget{septu}{Septu --- антифаговая система, выявленная в нетоксигенных штаммах}  
    \item \hypertarget{tcpA}{\textit{tcpA} --- ген, кодирующий субъединицу токсин-корегулируемых пилей}  
    \item \hypertarget{vps}{\textit{vps} --- кластер генов, ответственных за биосинтез виброполисахарида}  
    \item \hypertarget{hapR}{\textit{hapR} --- регулятор quorum sensing, влияющий на синтез ЭПС и формирование ругозного фенотипа}
    \item \hypertarget{ICE_SXT}{SXT/R391 --- семейство интегрированных конъюгативных элементов, часто несущих гены резистентности}    
    \item \hypertarget{ctxB}{\textit{ctxB} --- ген субъединицы B холерного токсина}
    \item \hypertarget{Жгутиковые_кластеры}{Жгутиковые кластеры Flg и Fli --- генные кластеры, кодирующие белки жгутикового аппарата}
    \item \hypertarget{ЛПС}{ЛПС --- липополисахарид, компонент внешней мембраны грамотрицательных бактерий}
    \item \hypertarget{O-антиген}{O-антиген --- полисахаридный антиген, компонент липополисахарида, определяющий серогруппу}
    \item \hypertarget{Дефенсом}{Дефенсом --- Defensome, совокупность систем антифаговой защиты организма}
    \item \hypertarget{ДНК}{ДНК --- дезоксирибонуклеиновая кислота}
    \item \hypertarget{РНК}{РНК --- рибонуклеиновая кислота}
    \item \hypertarget{Транзиция}{Транзиция --- точечная мутация, обусловленная заменой одного пуринового основания на другое (аденина на гуанин или наоборот) или одного пиримидинового основания на другое (тимина на цитозин или наоборот)}
    \item \hypertarget{Трансверсия}{Трансверсия --- точечная мутация, обусловленная заменой пуринового основания (аденин, гуанин) на пиримидиновое (тимин, цитозин) и наоборот}
    \item \hypertarget{рРНК,тРНК}{рРНК, тРНК --- рибосомная РНК, транспортная РНК}
    \item \hypertarget{ЭПС}{ЭПС --- экзополисахарид, виброполисахарид}
    \item \hypertarget{R-форма}{R-форма --- форма липополисахарида с отсутствующим или сокращённым O-антигеном}
    \item \hypertarget{Эффлюксный_насос}{Эффлюксный насос --- белок, обеспечивающий активный вынос токсинов и антибиотиков из клетки}  
    \item \hypertarget{Фаг}{Фаг --- бактериофаг, вирус, инфицирующий бактерии}  
    \item \hypertarget{Протеомика}{Протеомика --- набор методов и технологий определения и анализа белков}  
    \item \hypertarget{Геномика}{Геномика --- совокупность методов исследования геномной структуры и функций организма}
    \item \hypertarget{Норма}{Норма, нормальные условия, стандартные условия --- в данной работе - условия культивирования бактерий: немодифицированная среда LB, 37°C, в течение 18 часов, в условиях отсутствия специфических стрессовых факторов, в частности контакта с бактериофагами}

\end{enumerate}

% Обязательно добавляем это в конце каждой секции, чтобы 
% обеспечить переход на новую страницу
\clearpage % Перечень сокращений и условных обозначений
    
    
    %%% Здесь введение
\anonsection{Введение}

    XXXXXXXXXXXXXXXXXXXXXXXXXXXXXXXXXXXXXXXXXXXXXXXXXXXXXXXXXXXXXXXXXXXXXXXXXXXXXXXXXXXXXXXXXXXXXXXXXXXXXXXXXXXXXXXXXXXXXXXXXXXXXXXXXXXXXXXXXXXX. XXXXXXXXXXXXXXXXXXXXXXXXXXXXXXXXXXXXXXXXXXXXXXXXXXXXXXXXXXXXXXXXXXXXXXXXXXXXXXXXXXXXXXXXXXXXXXXXXXXXXXXXXXXXXXXXXXXXXXXXXXXXXXXXXXXXXXXXXXXX.XXXXXXXXXXXXXXXXXXXXXXXXXXXXXXXXXXXXXXXXXXXXXXXXXXXXXXXXXXXXXXXXXXXXXXXXXXXXXXXXXXXXXXXXXXXXXXXXXXXXXXXXXXXXXXXXXXXXXXXXXXXXXXXXXXXXXXXXXXXX.XXXXXXXXXXXXXXXXXXXXXXXXXXXXXXXXXXXXXXXXXXXXXXXXXXXXXXXXXXXXXXXXXXXXXXXXXXXXXXXXXXXXXXXXXXXXXXXXXXXXXXXXXXXXXXXXXXXXXXXXXXXXXXXXXXXXXXXXXXXX.

\anonsubsection{Цели и задачи настоящей работы} 
    XXXXXXXXXXXXXXXXXXXXXXXXXXXXXXXXXXXXXXXXXXXXXXXXXXXXXXXXXXXXXXXXXXXXXXXXXXXXXXXXXXXXXXXXXXXXXXXXXXXXXXXXXXXXXXXXXXXXXXXXXXXXXXXXXXXXXXXXXXXX.  
    
    В задачи настоящей работы входит: 
    \begin{enumerate}
        \item XXXXXXXXXXXXXXXXXXXXXXXXXXXXXXXXXXXXXXXXXXXXXXXXXXXXXXXXXXXXXXX.
        \item YYYYYYYYYYYYYYYYYYYYYYYYYYYYYYYYYYYYYYYYYYYYYYYYYYYYYYYYYYYYYYYY. 
        \item VVVVVVVVVVVVVVVVVVVVVVVVVVVVVVVVVVVVVVVVVVVVVVVVVVVVVVVVVVVVVVVV.
    \end{enumerate}

\anonsubsection{Положения выносимые на защиту}

    \begin{enumerate}[label={}]
        \item 1. AAAAAAAAAAAAAAAAAAAAAAAAAAAAAAAAA.
        \item 2. BBBBBBBBBBBBBBBBBBBBBBBBBBBBBBBBB. 
        \item 3. CCCCCCCCCCCCCCCCCCCCCCCCCCCCCCCCC.
    \end{enumerate}

    
\anonsubsection{Научная новизна работы}
    \begin{enumerate}[label={}]
        \item 1. AAAAAAAAAAAAAAAAAAAAAAAAAAAAAAAAA.
        \item 2. BBBBBBBBBBBBBBBBBBBBBBBBBBBBBBBBB. 
        \item 3. CCCCCCCCCCCCCCCCCCCCCCCCCCCCCCCCC.
    \end{enumerate}

% Обязательно добавляем это в конце каждой секции, чтобы 
% обеспечить переход на новую страницу
\clearpage % Введение
    %%% Литературный обзор %%%
\section{Литературный обзор}

\subsection{Подраздел 1}    
    
    Генетический материал холерного вибриона состоит из двух хромосом, может содержать плазмиды. Суммарный размер хромосом оценивается в 4 миллиона пар оснований. Обе хромосомы кодируют элементы, необходимые для жизнедеятельности клетки, например рРНК опероны, имеют постоянный размер и структуру, встречаются во всех представителях таксона.  
    
\subsection{Подраздел 2}

    Всемирная организация здравоохранения рекомендовала использовать антибиотические средства для лечения только тяжелых случаев холеры, и в современной стратегии борьбы с заболеванием массовое использование антибиотиков не рекомендуется \cite{book_sample}. Тем не менее, в ряде случаев антибиотики продемонстрировали свою эффективность, а потому стали применяться, наряду с регидратационными растворами, как основное средство в борьбе этим заболеванием \cite{repWHO2025}. Неизбежным итогом широкого применения антибиотиков против бактериальных инфекционных агентов становится эволюция и приспособление последних к применяемым против них препаратам \cite{CDC_AR_2019}. 

\subsection{Подраздел 3}
    
    Применение бактериофагов в медицине известно с начала XX века. С момента их открытия фаги успешно использовались для терапии инфекций, вызванных различными патогенными бактериями, включая \textit{Staphylococcus spp.}, \textit{Streptococcus spp.} и \textit{Pseudomonas aeruginosa}. Эффективность бактериофаги продемонстрировали против возбудителей дизентерии (\textit{Shigella spp.}), холеры (\textit{Vibrio cholerae}), брюшного тифа (\textit{Salmonella typhi}), а также для лечения раневых инфекций в военно-полевых условиях \cite{article_sample1}. 

    \subsubsection{Секция 1 внутри подраздела 3}
    
     Однако, широкое применение такого подхода ограничивается недостаточной изученностью молекулярных механизмов работы ряда антифаговых систем. 
    
\subsection{Подраздел 4}
    \subsubsection{Секция 1 внутри подраздела 4} 
    
    Описанный подход подразумевает цикл разнообразных экспериментальных и вычислительных работ. 

    \subsubsection{Секция 2 внутри подраздела 4}

    Вычисление CAI включает формирование эталонного набора генов. 

 %   Добавим несколько картинок (Рисунок \ref{fig:01-example-1}). 
 %   
 %   \addtwoimghere{01-example-1.png}{01-example-2.png}{0.49}{0.49}{В картинках также работают ссылки. Пусть \cite{newton2014newton}.}{fig:01-example-1}
 %   
 %   На приложение в тексте обязательно должна быть сделана ссылка ---  \hyperlink{app-a}{Приложение А}.
    
% Обязательно добавляем это в конце каждой секции, чтобы 
% обеспечить переход на новую страницу
\clearpage % Литературный обзор
    %%% Материалы и методы %%%
\section{Материалы и методы}
    В рамках экспериментальной части работы было проведено выращивание чистой культуры, тотальное выделение ДНК.
    
\subsection{Микробиология}
     Для культивирования и проведения дальнейшей работы были выбраны 5 штаммов. 
    
     \begin{table}[H]
           \caption{Штаммы} \label{table:01-strains}
           \begin{tabular}{|p{2.8cm}|p{1.8cm}|p{3.2cm}|p{3.8cm}|p{3cm}|}
           \hline Наименование & Год & Место \\
           \hline АФФ & 2011 & Калмыкия \\
           \hline МАА & 2016 & Москва \\
           \hline ИМА & 1970 & Саратов  \\
           \hline
           \end{tabular}
    \end{table}
     
     Выращивание чистой культуры проводили с использованием твёрдой агаризованной не модифицированной среды LB при температуре 37°C, pH=7.4 в течение 18 часов.  

\subsection{Геномика}

    Выделение ДНК проводили стандартным методом \cite{my20093}. 
    
    Подготовка библиотек геномной ДНК бактерий для секвенирования осуществлялась по протоколу Rapid Barcoding Kit 96 V14 (SQK-RBK114.96) с использованием соответствующих реагентов, рекомендованных в протоколе. Секвенирование геномной ДНК проводилось с использованием прибора PromethION компании Oxford Nanopore Technologies. 

\subsection{Анализ данных}

    Оценка качества полученных в результате секвенирования прочтений проводилась с использованием fastQC (v0.12.1) \cite{fastQC_2010}, Nanoplot (v1.42.0) \cite{nanoplot2023}, pycoQC (v2.5.0.3) \cite{pycoQC2019}. 
    
% \subsection{Первый раздел}
%   Сюда добавим какую-нибудь таблицу (Таблица \ref{table:01-coeffs}). 
%   
%   \begin{table}[H]
%       \caption{Таблица коэффициентов} \label{table:01-coeffs}
%       \begin{tabular}{|p{0.6cm}|p{4.9cm}|p{4.5cm}|p{4cm}|}
%       \hline \# & Колонка 1 & Колонка 2 & Колонка 3 \\
%       \hline 1 & Один & $f(x) + c$ & $4.1 $ \\
%       \hline 2 & Два & $f(x) - a$ & $4.2 $ \\
%       \hline 3 & Три & $f(x) \ \sim \ b$ & $4.3$ \\
%       \hline
%       \end{tabular}
%   \end{table}
%   
%   \subsubsection{Первый подраздел}
%       Первый подраздел первый подраздел первый подраздел первый подраздел первый подраздел первый подраздел первый подраздел первый подраздел первый подраздел первый подраздел первый подраздел первый подраздел первый подраздел 
%

% Обязательно добавляем это в конце каждой секции, чтобы 
% обеспечить переход на новую страницу
\clearpage % Материалы и методы
    %%% Результаты %%%
% Preamble file

\section{Результаты}

\subsection{Сборка генома}
    В работу были переданы 5 штаммов, культивированные в течение 18 часов на немодифицированной среде LB при 37°C. В результате по данным секвенирования с использованием ONT-прочтений были получены сборки последовательностей геномов, что согласуется с данными литературы (Таблица \ref{table:01-genome_stat}). 
    
        \begin{table}[H]
        \caption{Результаты сборки геномов} \label{table:01-genome_stat}
        \begin{tabular}{|p{2.8cm}|p{2.8cm}|p{2.8cm}|p{2.8cm}|p{2.8cm}|}
        \hline Наименование & Суммарная длина, bp & Хромосома 1, bp & Хромосома 2, bp & Плазмида, bp \\
        \hline S1 & 4 109 167 & 2 980 553 & 1 129 214 & - \\
        \hline S2 & 4 938 774 & 2 912 449 & 1 191 583 & 834 742 \\
        \hline S3 & 4 064 585 & 2 993 592 & 1 070 993 & - \\
        \hline S4 & 4 113 653 & 3 053 066 & 1 060 587 & - \\
        \hline S5 & 4 091 768 & 3 042 768 & 1 049 000 & - \\
        \hline
        \end{tabular}
        \end{table}
    
    Каждая сборка состоит из двух или трёх замкнутых кольцевых последовательностей (Рисунок \ref{fig:assemblies_fig}) и не содержит крупных недосеквенированных участков генома, что подтверждается данными проверки целостности сборки с использованием BUSCO - количество обнаруженных в полном объёме групп ортологов составляет от 99.5 \%. 

    \begin{figure}
        \centering
        \includegraphics[width=1\linewidth]{pattern//img/01-example-1.png}
        \caption{Состав собранных геномов}
        \label{fig:assemblies_fig}
    \end{figure}

    \subsubsection{Секция 1 Подраздела Сборка генома}
        Для более полной картины сравнения следует отметить, ген \textit{flgN}, в референсе не представленный. 

%    Добавим несколько картинок (Рисунок \ref{fig:01-example-1}). 
%    \addtwoimghere{01-example-1.png}{01-example-2.png}{0.49}{0.49}{В картинках также работают ссылки. Пусть \cite{newton2014newton}.}{fig:01-example-1} 
%    На приложение в тексте обязательно должна быть сделана ссылка ---  \hyperlink{app-a}{Приложение А}.

% Обязательно добавляем это в конце каждой секции, чтобы 
% обеспечить переход на новую страницу
\clearpage
 % Результаты эксперимента 
    %%% Выводы %%%
\section{Выводы}
% Сторонние функции для экспрессирующихся афз по данным литературы
\subsection{Геномная организация}
    
    Геномы большинства исследованных штаммов имеют характерную для вида организацию (2 кольцевые хромосомы) и суммарную длину около 4 млн пар оснований, что полностью соответствует современным представлениям о структуре генома этой бактерии. 
    
\subsection{Генетическое типирование}

    Типирование по ключевым генам полностью соответствует данным литературы и указывает на присутствие в выборке разнообразных по геномному составу вариантов патогена, характерных для разных волн пандемии. 

\subsection{Общие выводы и перспективы}

    Полученные результаты демонстрируют высокую степень геномного и функционального разнообразия штаммов как из клинических, так и из природных источников. Это подтверждает важность комплексного подхода к изучению эпидемиологии, эволюции и механизмов защиты возбудителя. 
    Полученные данные могут быть использованы для дальнейших исследований по микробиологии.
    
% Обязательно добавляем это в конце каждой секции, чтобы 
% обеспечить переход на новую страницу
\clearpage % Выводы
    \anonsection{Заключение}
    
    Исследование показало, что современные штаммы обладают сложной и динамичной системой геномной организации, включающей разнообразные механизмы защиты и адаптации.

% Обязательно добавляем это в конце каждой секции, чтобы 
% обеспечить переход на новую страницу
\clearpage % Заключение
    
    \input{pattern/inc/0-bibliography} % Подключение библиографии
    
    \appsection{Приложение 1} \hypertarget{app-a}{\label{app-a}}

\centering{Полногеномные выравнивания против референсного генома}

    \begin{figure}[h!]
        \centering
        \includegraphics[width=1\linewidth, height=0.8\textheight, keepaspectratio]{pattern/img/01-example-1.png}
        \captionsetup{name={}, labelformat=empty, skip=10pt} % Меняем "Рисунок" на "Приложение"
        \caption{Приложение 1 - Схемы выравниваний: \small a - штамм S1; b - штамм S2; c - штамм S3; d - штамм S4; e - штамм S5}
        \label{app-a:strain_ref_alignment}
    \end{figure}

\clearpage % Подключение приложения
    \appsection{Приложение 2} \hypertarget{app-b}{\label{app-b}}

\centering{}

    \begin{table}[H]
    \centering
        \captionsetup{name={}, labelformat=empty, skip=10pt}
        \caption{Приложение 2 - Большая таблица с инфой с RefSeq NCBI.} 
        \label{table:02-typing_res}
        \begin{tabular}{|p{2.8cm}|p{2.5cm}|p{3.5cm}|p{4.5cm}|}
        \hline Вариант аллеля ctxB & Штамм референса & NCBI accession number & Характеристика \\
        \hline  \textit{ctxB7} & 186 & GCF\_003015005.1 & геновариант El Tor, 2011, Украина, внешняя среда \\
        \hline  \textit{ctxB3} & 123AZ & GCF\_004358225.1 & типичный El Tor, 1977, Азербайджан, человек \\
        \hline  \textit{ctxB} & 157D & GCF\_004358245.1 & геновариант El Tor, 1993, Дагестан, человек \\
        \hline  \textit{ctxB1} & 41D & GCF\_003130475.1 & геновариант El Tor, 1998, Дагестан, человек \\
        \hline  \textit{ctxB7} & RND19191 & GCF\_000710455.1 & геновариант El Tor, 2010, Мск, человек (Ogawa) \\
        \hline  \textit{ctxB} & RND18826 & GCF\_000500735.1 & геновариант El Tor, 2005, Тверь, человек (Inaba) \\
        \hline  \textit{ctxB3} & N16961 & GCF\_900205735.1 & типичный El Tor \\
        \hline  \textit{ctxB} & Dakka35 & GCF\_000961975.1 & типичный Classic, 1958, Пакистан, человек \\
        \hline  \textit{ctxB3} & M818 & GCF\_000966385.1 & типичный El Tor, 1970, Балаково, человек \\
        \hline
        \end{tabular}
        \end{table} 


\clearpage

 % Подключение приложения
        \appsection{Приложение 3} \hypertarget{app-b1}{\label{app-b1}}

\centering{Результаты типирования изучаемых штаммов \textit{V.cholerae} по регионам \textit{ctxB} и \textit{tcpA}, кодирующим основные факторы патогенности}

    \begin{figure}[h!]
        \centering
        \includegraphics[width=1.2\linewidth, height=0.8\textheight, keepaspectratio, angle=90]{pattern/img/01-example-1.png}
        \captionsetup{name={}, labelformat=empty, skip=10pt} % Меняем "Рисунок" на "Приложение"
        \caption{Приложение 3 - Филогенетические деревья.}
        \label{app-b1:phylo_trees_ctxBtcpA}
    \end{figure}


\clearpage % Подключение приложения

\end{document}
